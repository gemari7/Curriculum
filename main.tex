\documentclass[11pt,a4paper]{article}
\usepackage[utf8]{inputenc}
\usepackage[T1]{fontenc}
% pictures
\usepackage{graphicx}
\pagestyle{empty} %Limpia la pagina
% update default paragraph indent, and header space
\setlength{\topskip}{0pt}      % between header and text (0 needed for vertical centring)
\usepackage{parskip}           % remove paragraph indents
%Para las macros
\usepackage{xparse}
% links
\usepackage[colorlinks=true,linkcolor=black,citecolor=cyan, filecolor=magenta,urlcolor=blue,menucolor=cyan ]{hyperref}%Hipervínculos, da color, etc.
%-------------------------------------------------------------------------------
%                         Iconos
%-------------------------------------------------------------------------------

% icon font
\usepackage{fontawesome} 

%-------------------------------------------------------------------------------
%                         Margenes y Medidas
%-------------------------------------------------------------------------------
\usepackage{calc}
%-----------MARGEN
\newlength\cvMargin
\setlength\cvMargin{1cm}

\newlength\cvSideWidth
\setlength\cvSideWidth{0.3\paperwidth-\cvMargin}

\newlength\cvPictureWidth
\setlength\cvPictureWidth{4cm}

\newlength\cvLanguageBarWidth
\setlength\cvLanguageBarWidth{5em}

\newlength\cvLanguageBarHeight
\setlength\cvLanguageBarHeight{0.75em}

\newlength\cvTimeDotSep
\setlength\cvTimeDotSep{0.4cm}

\newlength\cvHeaderIconWidth
\setlength{\cvHeaderIconWidth}{\maxof{\widthof{\faBriefcase}}{\widthof{\faGraduationCap}}}

\newlength\cvMainWidth
\setlength\cvMainWidth{\paperwidth-4\cvMargin-\cvSideWidth}

\newlength{\cvSectionSep}
\setlength{\cvSectionSep}{0.4cm}

%Dimensiones de PaginaNormal
\usepackage[margin=\cvMargin,noheadfoot,a4paper]{geometry}




%-------------------------------------------------------------------------------
%                        Estilos
%-------------------------------------------------------------------------------
\usepackage{tikz}
%La calc biblioteca permite utilizar combinaciones complejas de nodos (por ejemplo, fracciones de coordenadas de nodo)
% la positioning biblioteca permite un posicionamiento relativo conveniente de los nodos
\usetikzlibrary{calc,positioning,backgrounds,matrix}

\tikzset{
	contactIcon/.style={%
	    minimum height=\baselineskip,
	},
	  contactText/.style={%
	    minimum height=\baselineskip,
	    text depth=0pt,
	},
	languageText/.style={},
	progressArea/.style={%
		draw,
		rectangle,
		minimum width=\cvLanguageBarWidth,
		minimum height=\cvLanguageBarHeight,
		cvProgressBar},
	progressBar/.style={%
		minimum height=\cvLanguageBarHeight,
		rectangle,
		draw,
		fill,
		cvProgressBar,
		anchor=west},
	headerIcon/.style={
		minimum width=\cvHeaderIconWidth,
		anchor=center,
	},
	sectionTitle/.style={
		anchor=north west,
		align=left},
	itemText/.style = {
		text width=\cvMainWidth-\cvTimeDotSep,
		black,
		anchor=north west,
	},
	invisibletimedot/.style = {
		circle,
		minimum width=3pt,
		anchor=center
	},
	timedot/.style = {
		invisibletimedot,
		draw,
		fill,
		black,
	},
}

%-------------------------------------------------------------------------------
%                         Colores
%-------------------------------------------------------------------------------

\definecolor{cvColorLateral}{HTML}{ABEBC6}
\definecolor{cvAccent}{HTML}{474A65}
\definecolor{cvBarra}{HTML}{357F2D}
\definecolor{cvProgressBar}{HTML}{357F2D}




   

  
   


%-------------------------------------------------------------------------------
%                         Macro Barra de los titulos (Lateral)
%-------------------------------------------------------------------------------
% based on https://tex.stackexchange.com/questions/65731
\makeatletter
\def\cv@hrulefill{{\color{cvBarra}\leavevmode\leaders\hrule height 1pt\hfill\kern\z@}}

% line before and after text (some tweaking is required here)
% based on https://tex.stackexchange.com/questions/15119
\NewDocumentCommand{\ruleline}{m}{\par\noindent\raisebox{\baselineskip/4}
{\makebox[\linewidth]{\cv@hrulefill\hspace{1ex}\raisebox{-\baselineskip/4}
{#1}\hspace{1ex}\cv@hrulefill}}}
\makeatother
%-------------------------------------------------------------------------------
%                         Macro Section
%-------------------------------------------------------------------------------
\newcommand{\cvSection}[1]{\Large\textbf{#1}}
%-------------------------------------------------------------------------------
%                         Macro Item
%-------------------------------------------------------------------------------

\newcommand{\cvItem}[4]{
	\parbox[t]{0.2\textwidth}{ #1} 
	\parbox[t]{0.81\textwidth}{\textbf{#2}\\ \textit{#3}\\#4}

}
%\parbox[position]{width}{text}
%Titulo
%Lugar    Fecha
%Contenido
\begin{document}

    

	
%Vertical
\vspace*{\fill}

    %Rectangulo----------------------------------------------------------
    \begin{tikzpicture}[remember picture,overlay]
        \fill[cvColorLateral] (current page.north west) rectangle ++(\cvSideWidth+2\cvMargin,-\paperheight);
    \end{tikzpicture}
    %---------------------------------------------------------------------
    
    \begin{minipage}{\cvSideWidth}
        \begin{center}
        %-------------------------------------------------------------------------------
        %                         Seccion de Identificacion
        %-------------------------------------------------------------------------------
        %Foto
        \begin{tikzpicture}
          \node[
            circle,
            minimum size=\cvPictureWidth,
            path picture={
              \node at (path picture bounding box.center){
              \includegraphics[width=\cvPictureWidth]{img/profile2.jpg}
              };
            }]
            {};
        \end{tikzpicture}
        {\LARGE
        Gema María\\
        \vspace{0.1cm}
        García Ortega}
        \vspace{0.5cm}
        
        {\color{cvAccent} Ingeniero Informático}
        
        \vspace{0.5cm}
        \ruleline{Sobre mi}
        Soy una persona asertiva, trabajadora, con gran juicio crítico-analítico. Me gusta afrontar nuevos retos y trabajar en equipo.
        
        \vspace{4pt}
        %-------------------------------------------------------------------------------
        %                         Sección de contacto
        %-------------------------------------------------------------------------------
        \ruleline{Datos personales}
          \vspace{4pt}
  
          \begin{tikzpicture}[every node/.style={inner sep=0pt, outer sep=0pt}]
          \matrix [
            column 1/.style={anchor=center,contactIcon},
            column 2/.style={anchor=west,align=left,contactText},
            column sep=5pt,
            row sep=5pt] (contact) {
            
                \node{\faMapMarker}; 
                  & \node{Calle Ronda de la Manca, 51};\\
                \node{\faEnvelope}; 
                  & \node{\href{mailto:gemari777@gmail.com}{gemari777@gmail.com}};\\
                \node{\faPhone}; 
                  & \node{+34 957 48 51 87 \\680 95 87 99};\\
                \node{\faGlobe}; 
                  & \node{\href{https://johndoe.com}{johndoe.com}};\\
                \node{\faGithub}; 
                  & \node{\href{https://github.com/johndoe}{johndoe}};\\
                \node{\faLinkedinSquare}; 
                  & \node{\href{https://www.linkedin.com/in/johndoe/}{johndoe}};\\
                \node{\faTwitter}; 
                  & \node{\href{https://twitter.com/JohnDoe}{@JohnDoe}};\\
                \node{\faKey}; 
                  & \node{\href{https://keybase.io/johndoe}{\texttt{EA31 B617 B3A1 EFF0}}};\\
        };
\end{tikzpicture}
          
          \vspace{4pt}
        
        \end{center}
    \end{minipage}
\vspace*{\fill}
 
	\begin{tikzpicture}[
	every node/.style={
		inner sep=0pt,
		outer sep=0pt
	},
	remember picture,
	overlay,
	shift={($(current page.north west)%
		+(\cvSideWidth+3\cvMargin+\cvTimeDotSep,-\cvMargin)$)}]
	
	
	
	
%-------------------------------------------------------------------------------
%                         Seccion de Educación
%-------------------------------------------------------------------------------
	
	\node[ %---------------TITULO
		sectionTitle] 
		at (0,0) (title 1) %ID
		{\cvSection{Estudios}};%NOMBRE
	\node[ %---------------ICONO
		left=\cvTimeDotSep of title 1,headerIcon] 
		{\faGraduationCap};
	%----------------------LINEA
	\begin{scope}[on background layer]
		\draw[line width=2pt,cvBarra] 
		let \p1=(title 1.south west), 
			\p2=(current page.east) in 
		(\x1,\y1-0.1cm) to (\x2,\y1-0.1cm);
	\end{scope}
	
	%-----------------------------------------
	%                  Item 1
	%-----------------------------------------
	\node[
		below=\cvSectionSep of title 1.south west,
		itemText] 
		(item 1 header) 
		{\phantom{Item}};
	\node[
		below=\cvSectionSep of title 1.south west,
		itemText] 
		(item 1) 
		{\cvItem{2015 - 2020}%
			{Grado en Ingeniería Informática}%
			{Universidad de Córdoba. }%
			{Matrícula de honor Trabajo Fin de Grado, titulado  \href{https://github.com/gemari7/AplicacionMovilParaLaGestionDeLaProteccionContraIncendios}{Aplicación móvil para la gestión de la protección contra incendios}.\\
			Matrícula de honor en Auditoria Informática.\\			
			Matrícula de honor en Matemáticas aplicadas a la Computación.\\
			Matrícula de honor en Arquitectura de Computadores.}%
		};
	\node[
		left=\cvTimeDotSep of item 1 header,
		timedot] 
		(start) 
		{};
	%-----------------------------------------
	%                  Item 2
	%-----------------------------------------
	\node[
		below=\cvSectionSep of item 1.south west,
		itemText] 
		(item 2 header) 
		{\phantom{Item}};
	\node[
		below=\cvSectionSep of item 1.south west,
		itemText] 
		(item 2) 
		{\cvItem{2013 - 2015}%
			{Bachillerato Tecnologico}%
			{IES Maimónides. }%
			{Nota media notable.}%
		};
	\node[
		left=\cvTimeDotSep of item 2 header,
		timedot] 
		(end) 
		{};
	\node[
		left=\cvTimeDotSep of item 2.south west,
		invisibletimedot] 
		(end) 
		{};
		\draw (start) to (end.center);

	
%-------------------------------------------------------------------------------
%                         Seccion de Formación complementaria
%-------------------------------------------------------------------------------

	\node[ %---------------TITULO
		below=0.6cm of item 2.south west,
		sectionTitle] 
		(title 2) 
		{\cvSection{Formación Complementaria}};
	\node[ %---------------ICONO
		left=\cvTimeDotSep of title 2,headerIcon] 
		{\faWpforms};
	\node[
		below=0.6cm of item 2.south west,
		sectionTitle] 
		(title 2 dummy) 
		{\phantom{\cvSection{Education}}};
	%--------------------LINEA
	\begin{scope}[on background layer]
		\draw[line width=2pt,cvBarra] 
		let \p1=(title 2 dummy.south west), 
			\p2=(current page.east) in 
		(\x1,\y1-0.1cm) to (\x2,\y1-0.1cm);
	\end{scope}
	
	%-----------------------------------------
	%                  Item 1
	%-----------------------------------------
	\node[
		below=\cvSectionSep of title 2.south west,
		itemText] 
		(item 1 header) 
		{\phantom{Item}};
	\node[
		below=\cvSectionSep of title 2.south west,
		itemText] 
		(item 1) 
		{\cvItem{2019}%
			{Certificado de manipulador de alimentos de mayor riesgo}%
			{ALARCOR D.L. }%
			{Certificado de acreditación de la formación correspondiente a manipulador de alimentos de mayor riesgo.}%
		};
	\node[
		left=\cvTimeDotSep of item 1 header,
		timedot] 
		(start) 
		{};
	%-----------------------------------------
	%                  Item 2
	%-----------------------------------------
	\node[
		below=\cvSectionSep of item 1.south west,
		itemText] 
		(item 2 header) 
		{\phantom{Item}};
	\node[
		below=\cvSectionSep of item 1.south west,
		itemText] 
		(item 2) 
		{\cvItem{2018}%
			{Certificado de CCNA 2}%
			{Universidad de Córdoba. }%
			{CCNA Routing and Switching: Principios básicos de routing y switching.}%
		};
	\node[
		left=\cvTimeDotSep of item 2 header,
		timedot] 
		{};
	%-----------------------------------------
	%                  Item 3
	%-----------------------------------------
	\node[
		below=\cvSectionSep of item 2.south west,
		itemText] 
		(item 3 header) 
		{\phantom{Item}};
	\node[
		below=\cvSectionSep of item 2.south west,
		itemText] 
		(item 3) 
		{\cvItem{2018}%
			{Certificado de CCNA 1}%
			{Universidad de Córdoba. }%
			{CCNA Routing and Switching: Introducción a redes.}%
		};
	\node[
		left=\cvTimeDotSep of item 3 header,
		timedot] 
		{};
	%-----------------------------------------
	%                  Item 4
	%-----------------------------------------
	\node[
		below=\cvSectionSep of item 3.south west,
		itemText] 
		(item 4 header) 
		{\phantom{Item}};
	\node[
		below=\cvSectionSep of item 3.south west,
		itemText] 
		(item 4) 
		{\cvItem{2018}%
			{Competencia en Kaggle}%
			{Universidad de Córdoba. }%
			{Certificado del grupo de Investigación de Aprendizaje y Redes Neuronales Artificiales (AYRNA)\\ \href{https://www.kaggle.com/c/competition-iaa-1819}{https://www.kaggle.com/c/competition-iaa-1819}.}%
		};
	\node[
		left=\cvTimeDotSep of item 4 header,
		timedot] 
		{};
	\node[
		left=\cvTimeDotSep of item 4.south west,
		invisibletimedot] 
		(end) 
		{};
		\draw (start) to (end.center);
	

	


%-------------------------------------------------------------------------------
%                         Seccion de Experiencia Profesional
%-------------------------------------------------------------------------------
	
	\node[ %---------------TITULO
		below=0.6cm of item 4.south west,
		sectionTitle] 
		(title 3) 
		{\cvSection{Experiencia Profesional}};
	\node[ %---------------ICONO
		left=\cvTimeDotSep of title 3,headerIcon] 
		{\faBriefcase};
	\node[
		below=0.6cm of item 4.south west,
		sectionTitle] 
		(title 3 dummy) 
		{\phantom{\cvSection{Formación Complementaria}}};
		 %---------------LINEA
	\begin{scope}[on background layer]
		\draw[line width=2pt,cvBarra] 
		let \p1=(title 3 dummy.south west), 
			\p2=(current page.east) in 
		(\x1,\y1-0.1cm) to (\x2,\y1-0.1cm);
	\end{scope}
	
	%-----------------------------------------
	%                  Item 1
	%-----------------------------------------
	\node[
		below=\cvSectionSep of title 3.south west,
		itemText] 
		(item 1 header) 
		{\phantom{Item}};
	\node[
		below=\cvSectionSep of title 3.south west,
		itemText] 
		(item 1) 
		{\cvItem{2015 - 2020}%
			{Personal de barra}%
			{Estadio del Arcángel. }%
			{Servir en la barra y atender al público.}%
		};
	\node[
		left=\cvTimeDotSep of item 1 header,
		timedot] 
		(start)
		{};
	\node[
		left=\cvTimeDotSep of item 1.south west,
		invisibletimedot] 
		(end) 
		{};
	\draw (start) to (end.center);
	

	
	
	
	
	
	
\end{tikzpicture}











\end{document}
